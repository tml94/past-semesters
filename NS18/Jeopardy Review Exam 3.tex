\documentclass{beamer}
\setbeamertemplate{navigation symbols}{}
\setbeamersize{text margin left=0cm, text margin right=0cm}
\usepackage{tcolorbox}
    \tcbuselibrary{skins}
\usepackage{ocgx}
\usepackage{enumitem}


\newtcolorbox{InnerSubjectBox}{
    enhanced,
    nobeforeafter,
    arc=0pt,
    width=.2\paperwidth,
    boxrule=.4pt,
    colframe=white,
    center upper,
    interior style={
        top color=blue,
        bottom color=black
    },
    colupper=white,
}

\newcommand{\subjects}[5]{×
    \begin{InnerSubjectBox}
        #1
    \end{InnerSubjectBox}%
    \begin{InnerSubjectBox}
        #2
    \end{InnerSubjectBox}%
    \begin{InnerSubjectBox}
        #3
    \end{InnerSubjectBox}%
    \begin{InnerSubjectBox}
        #4
    \end{InnerSubjectBox}%
    \begin{InnerSubjectBox}
        #5
    \end{InnerSubjectBox}
}

\newtcolorbox{InnerPrizeBox}{
    enhanced,
    nobeforeafter,
    arc=0pt,
    width=.2\paperwidth,
    boxrule=.4pt,
    colframe=white,
    center upper,
    interior style={
        top color=blue,
        bottom color=black
    },
    colupper=white,
    valign=center,
    fontupper=\LARGE\bfseries,
    height=.176\paperheight
}

\newcommand{\prizes}{%
    \foreach \i in {1,2,...,5}{%
        \begin{InnerPrizeBox}%
            \begin{ocg}{s\i-100}{s\i-100}{1}{\hyperlink{s\i-100}{100}}\end{ocg}
        \end{InnerPrizeBox}%
        }%

    \foreach \i in {1,2,...,5}{%
        \begin{InnerPrizeBox}%
            \begin{ocg}{s\i-200}{s\i-200}{1}{\hyperlink{s\i-200}{200}}\end{ocg}
        \end{InnerPrizeBox}%
        }%

    \foreach \i in {1,2,...,5}{%
        \begin{InnerPrizeBox}%
            \begin{ocg}{s\i-300}{s\i-300}{1}{\hyperlink{s\i-300}{300}}\end{ocg}
        \end{InnerPrizeBox}%
        }%

    \foreach \i in {1,2,...,5}{%
        \begin{InnerPrizeBox}%
            \begin{ocg}{s\i-400}{s\i-400}{1}{\hyperlink{s\i-400}{400}}\end{ocg}
        \end{InnerPrizeBox}%
        }%

    \foreach \i in {1,2,...,5}{%
        \begin{InnerPrizeBox}%
            \begin{ocg}{s\i-500}{s\i-500}{1}{\hyperlink{s\i-500}{500}}\end{ocg}
        \end{InnerPrizeBox}%
        }%
    }

\newtcolorbox{QuestionHeadFoot}[1][]{
    enhanced,
    before=\vskip-.7ex,
    after=,
    arc=0pt,
    width=\paperwidth,
    boxrule=.4pt,
    colframe=white,
    center upper,
    center lower,
    interior style={
        top color=blue,
        bottom color=black
    },
    colupper=white,
    collower=white,
    valign=center,
    fontupper=\LARGE\bfseries,
    fontlower=\LARGE\bfseries,
    height=.176\paperheight,
    sidebyside,
    segmentation style={white,solid,line width=.4pt},
    #1
}

\newcommand{\header}[2]{
    \begin{QuestionHeadFoot}
         #1 \tcblower #2
    \end{QuestionHeadFoot}
}

\newcommand{\footer}[1]{
    \begin{QuestionHeadFoot}[before=\vskip-2.7ex]
        \hyperlink{question#1}{Question} \\ \hyperlink{answer#1}{Answer} \tcblower
        \hideocg{#1}{Done!} \\ \hyperlink{home}{Home}
    \end{QuestionHeadFoot}
}

\newcommand{\content}[4]{
\begin{frame}
    \hypertarget<1>{answer#1}{}
    \hypertarget<2>{question#1}{}
    \hypertarget{#1}{}
    \header{#2}{#3}
  #4
    \footer{#1}
\end{frame}
}

\newtcolorbox{textarea}[1][]{
    nobeforeafter,
    height=6.2cm,
    boxrule=0pt,
    center upper,
    valign=center,
    #1
}

\begin{document}
\begin{frame}
\hypertarget{home}{}
\vspace*{-.5cm}
    \subjects{Rationals}{Exponential}{Logarithm}{Log Prop}{Grab Bag}
    \prizes
\end{frame}

\content                       % 4 arguments
    {s1-100}                     % question internal identifier
    {Rational}                          % subject number
    {100}{                       % question prize
        \begin{textarea}[]         % question/answer content
        \only<1>{                  % question content
            What is the domain of $f(x)=\frac{x^2-16}{x^2-x-6}$?
        }
        \only<2>{                  % answer content
            $(-\infty,-2)\cup(-2,3)\cup(3,\infty)$
        }
        \end{textarea}
    }
\content                       % 4 arguments
    {s1-200}                     % question internal identifier
    {Rational}                          % subject
    {200}{                       % question prize
        \begin{textarea}[]         % question/answer content
        \only<1>{                  % question content
            What are the zeros of $f(x)=\frac{x^2-25}{x^2+x-6}$?
        }
        \only<2>{                  % answer content
            $(5,0)$ and $(-5,0)$
        }
        \end{textarea}
    }  
    
\content                       % 4 arguments
    {s1-300}                     % question internal identifier
    {Rational}                          % subject
    {300}{                       % question prize
        \begin{textarea}[]         % question/answer content
        \only<1>{                  % question content
            Find all asymptotes of the equation $f(x)=\frac{5x-15}{x-4}$.
        }
        \only<2>{                  % answer content
            Horizontal: $y=5$\\ Vertical: $x=4$
        }
        \end{textarea}
    }   

\content                       % 4 arguments
    {s1-400}                     % question internal identifier
    {Rational}                          % subject
    {400}{                       % question prize
        \begin{textarea}[]         % question/answer content
        \only<1>{                  % question content
            Determine the equation for a rational function if the $x$-intercept of the function is $(4,0)$ , the $y-$intercept is $(0,-2)$, and the equations of the asymptotes are $y=-1$ and $x=2$.
        }
        \only<2>{                  % answer content
            $y=\frac{-x+4}{x-2}$
        }
        \end{textarea}
    }  
    
\content                       % 4 arguments
    {s1-500}                     % question internal identifier
    {Rational}                          % subject number
    {500}{                       % question prize
        \begin{textarea}[]         % question/answer content
        \only<1>{                  % question content
           A T-shirt manufacturer has found the cost of running their business to be \$6 per T-shirt and has overhead costs of \$1,300. Write a function that represents the average cost for producing $x$ T-shirts.
        }
        \only<2>{                  % answer content
            $y=\frac{6x+1300}{x}$
        }
        \end{textarea}
    }   

\content                       % 4 arguments
    {s2-100}                     % question internal identifier
    {Exponential}                          % subject number
    {100}{                       % question prize
        \begin{textarea}[]         % question/answer content
        \only<1>{                  % question content
            Let $f(x)=3 \cdot \left(\frac{5}{3}\right)^x + 3$. What is the range of $f(x)$?
        }
        \only<2>{                  % answer content
            $(3, \infty)$
        }
        \end{textarea}
    }    
    
\content                       % 4 arguments
    {s2-200}                     % question internal identifier
    {Exponential}                          % subject number
    {200}{                       % question prize
        \begin{textarea}[]         % question/answer content
        \only<1>{                  % question content
            How much money should you invest at 7.5\% compounded quarterly so that you have \$10,000 after 5 years?
        }
        \only<2>{                  % answer content
            $1,856.78$
        }
        \end{textarea}
    }
   
\content                       % 4 arguments
    {s2-300}                     % question internal identifier
    {Exponential}                          % subject number
    {300}{                       % question prize
        \begin{textarea}[]         % question/answer content
        \only<1>{                  % question content
            A piece of machinery, initially purchased for \$25,000 decreases in value by 1.4\% per year. Determine a model of the form $y=C\cdot b^x$ that can be used to predict the value of this machinery if $x$ is measured in years.
        }
        \only<2>{                  % answer content
            $y=25,000(0.986)^x$
        }
        \end{textarea}
    } 
    
\content                       % 4 arguments
    {s2-400}                     % question internal identifier
    {Exponential}                          % subject number
    {400}{                       % question prize
        \begin{textarea}[]         % question/answer content
        \only<1>{                  % question content
            A species of snake was introduced in an area 10 years ago. It is estimated that there are 3,500 snakes in the area now, and the population has a continuous exponential growth rate of 7\% per year. How many snakes will there be 20 years from now?
        }
        \only<2>{                  % answer content
            14,193 snakes will be present 20 years from now.
        }
        \end{textarea}
    }  
    
\content                       % 4 arguments
    {s2-500}                     % question internal identifier
    {Exponential}                          % subject number
    {500}{                       % question prize
        \begin{textarea}[]         % question/answer content
        \only<1>{                  % question content
            What is the doubling time for a population of rabbits that grows from 60 to 500 in 18 months?
        }
        \only<2>{                  % answer content
            5.885 months
        }
        \end{textarea}
    }   
    
\content                       % 4 arguments
    {s3-100}                     % question internal identifier
    {Logarithm}                          % subject number
    {100}{                       % question prize
        \begin{textarea}[]         % question/answer content
        \only<1>{                  % question content
            Write the following as a logarithm: \[245^\frac{1}{2}=x\]
        }
        \only<2>{                  % answer content
            $\log_{245}(x)=\frac{1}{2}$
        }
        \end{textarea}
    } 
                            
\content                       % 4 arguments
    {s3-200}                     % question internal identifier
    {Logarithm}                          % subject number
    {200}{                       % question prize
        \begin{textarea}[]         % question/answer content
        \only<1>{                  % question content
            What is the domain of $\log_3(x+11)$?.
        }
        \only<2>{                  % answer content
            $(-11, \infty)$
        }
        \end{textarea}
    }

\content                       % 4 arguments
    {s3-300}                     % question internal identifier
    {Logarithm}                          % subject number
    {300}{                       % question prize
        \begin{textarea}[]         % question/answer content
        \only<1>{                  % question content
            Solve the equation $3\log_2(x)=4$.
        }
        \only<2>{                  % answer content
            $x=2.52$
        }
        \end{textarea}
    }    

\content                       % 4 arguments
    {s3-400}                     % question internal identifier
    {Logarithm}                          % subject number
    {400}{                       % question prize
        \begin{textarea}[]         % question/answer content
        \only<1>{                  % question content
           A pork roast is removed from a freezer that is $22^\circ$F and placed in a room that is $77^\circ$F. The number of minutes that it takes the temperature of the roast to reach $x$ degrees Fahrenheit is given by the formula $T=\frac{100}{3}\ln\left(\frac{50}{77-x}\right)$. If 20 minutes have passed, what is the temperature of the pork roast?
        }
        \only<2>{                  % answer content
            The pork roast is 50 degrees Fahrenheit after 20 minutes have elapsed since the pork roast was removed from the freezer.
        }
        \end{textarea}
    }    

\content                       % 4 arguments
    {s3-500}                     % question internal identifier
    {Logarithm}                          % subject number
    {500}{                       % question prize
        \begin{textarea}[]         % question/answer content
        \only<1>{                  % question content
            Solve for $x$. \[ \frac{e^{x+5}}{e^{3x}}=e^{x-1}\]
        }
        \only<2>{                  % answer content
            $x=2$
        }
        \end{textarea}
    }
    
\content                       % 4 arguments
    {s4-100}                     % question internal identifier
    {Log Prop}                          % subject number
    {100}{                       % question prize
        \begin{textarea}[]         % question/answer content
        \only<1>{                  % question content
            Expand as far as possible $\ln\left(\frac{e^2}{3}\right)$
        }
        \only<2>{                  % answer content
            $2-\ln(3)$
        }
        \end{textarea}
    } 
    
\content                       % 4 arguments
    {s4-200}                     % question internal identifier
    {Log Prop}                          % subject number
    {200}{                       % question prize
        \begin{textarea}[]         % question/answer content
        \only<1>{                  % question content
            Write as a single logarithm: $\ln(x)+\frac{1}{3}\ln(27)-\ln(y)+\ln(z)$
        }
        \only<2>{                  % answer content
            $\ln\left(\frac{27^{\frac{1}{3}}xz}{y}\right)$
        }
        \end{textarea}
    }
    
\content                       % 4 arguments
    {s4-300}                     % question internal identifier
    {Log Prop}                          % subject number
    {300}{                       % question prize
        \begin{textarea}[]         % question/answer content
        \only<1>{                  % question content
            Expand as far as possible: $\log_5\left(\frac{ab^2}{5cd}\right)$
        }
        \only<2>{                  % answer content
            $\log_5(a)+2\log_5(b)-1-\log_5(c)-\log_5(d)$
        }
        \end{textarea}
    } 
    
\content                       % 4 arguments
    {s4-400}                     % question internal identifier
    {Log Prop}                          % subject number
    {400}{                       % question prize
        \begin{textarea}[]         % question/answer content
        \only<1>{                  % question content
            Write as a single logarithm: $\log_7(x)-\log_7(3)+4\log_7(y)-1$
        }
        \only<2>{                  % answer content
            $\log_7\left(\frac{xy^4}{21}\right)$
        }
        \end{textarea}
    }
    
\content                       % 4 arguments
    {s4-500}                     % question internal identifier
    {Log Prop}                          % subject number
    {500}{                       % question prize
        \begin{textarea}[]         % question/answer content
        \only<1>{                  % question content
            Solve the following for $x$ \[\log(2x+1)-\log(x-2)=1\]
        }
        \only<2>{                  % answer content
            $x=\frac{21}{8}$
        }
        \end{textarea}
    } 

\content                       % 4 arguments
    {s5-100}                     % question internal identifier
    {Grab Bag}                          % subject number
    {100}{                       % question prize
        \begin{textarea}[]         % question/answer content
        \only<1>{                  % question content
            Suppose a cost-benefit model is given by $T=f(x)=\frac{22x}{100-x}$, where $T$ is the time in minutes, to memorize $x$ random facts. Approximately how many facts can be memorized if a person studies for 30 minutes?\\
            \begin{enumerate}[label=\rm{(\alph*)}]
            	\item Less than 15 facts
            	\item Between 15 and 30 facts
            	\item Between 30 and 45 facts
            	\item Between 45 and 60 facts
            	\item More than 60 facts
            \end{enumerate}
        }
        \only<2>{                  % answer content
            $(d)$ Between 45 and 60 facts
        }
        \end{textarea}
    }
    
\content                       % 4 arguments
    {s5-200}                     % question internal identifier
    {Grab Bag}                          % subject number
    {200}{                       % question prize
        \begin{textarea}[]         % question/answer content
        \only<1>{                  % question content
            Determine a formula for the exponential function of the form $y=C\cdot b^x$ that passes through the points $(-1,3)$ and $(2,192)$.
            \begin{enumerate}[label=\rm{(\alph*)}]
            	\item $y=12 \cdot 4^x$
            	\item $y=3 \cdot 64^x$
            	\item $y= 3 \cdot 63^x$
            	\item $y=3 \cdot 8^x$
            \end{enumerate}

        }
        \only<2>{                  % answer content
            $(a)$ $y=12 \cdot 4^x$
        }
        \end{textarea}
    } 
    
\content                       % 4 arguments
    {s5-300}                     % question internal identifier
    {Grab Bag}                          % subject number
    {300}{                       % question prize
        \begin{textarea}[]         % question/answer content
        \only<1>{                  % question content
            Determine a formula for the inverse function, $f^{-1}(x)$ for $f(x)=2^{x-5}$.
            \begin{enumerate}[label=\rm{(\alph*)}]
            	\item $f^{-1}(x)=5^{x+2}$
            	\item $f^{-1}(x)=\log_2(x+5)$
            	\item $f^{-1}(x)=\log_2(x-5)$
            	\item $f^{-1}(x)=\log_2(x)+5$	
            	\item $f^{-1}(x)=2^{x+5}$
            \end{enumerate}

            
        }
        \only<2>{                  % answer content
            $(d)$ $f^{-1}(x)=\log_2(x)+5$
        }
        \end{textarea}
    }
    
\content                       % 4 arguments
    {s5-400}                     % question internal identifier
    {Grab Bag}                          % subject number
    {400}{                       % question prize
        \begin{textarea}[]         % question/answer content
        \only<1>{                  % question content
           Solve the equation: \[\log_9(x-3)+\log_9(2x+1)=1\]
           \begin{enumerate}[label=\rm{(\alph*)}]
           		\item $x=-\frac{3}{2}, 4$ only
           		\item $x=\frac{11}{3}$ only
           		\item $x=-\frac{1}{2}, 3$ only
           		\item $x=3$ only
           		\item $x=4$ only
           \end{enumerate}

        }
        \only<2>{                  % answer content
            $(e)$ $x=4$ only
        }
        \end{textarea}
    }  
    
\content                       % 4 arguments
    {s5-500}                     % question internal identifier
    {Grab Bag}                          % subject number
    {500}{                       % question prize
        \begin{textarea}[]         % question/answer content
        \only<1>{                  % question content
            The population of a species of bird grows from 1300 to 1840 in 6 years. Use the exponential growth model $A(t)=Pe^{kt}$ with $t$ measured in years, to determine the value of $k$. The value of $k$ is:
            \begin{enumerate}[label=\rm{(\alph*)}]
            	\item More than 0.063
            	\item Between 0.059 and 0.063
            	\item Between 0.055 and 0.059
            	\item Between 0.051 and 0.055
            	\item Less than 0.051
            \end{enumerate}

        }
        \only<2>{                  % answer content
            $(c)$ Between 0.055 and 0.059
        }
        \end{textarea}
    }                                         
        
\end{document}




























