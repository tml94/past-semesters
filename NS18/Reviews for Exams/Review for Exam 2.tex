\documentclass[frontgrid]{flacards}
\usepackage{color} 
\usepackage{graphicx}
\usepackage[notextcomp]{kpfonts} 
\usepackage{amsthm,amssymb,amsmath}
\usepackage{graphicx}
\usepackage{enumitem}
\usepackage{bm}
\usepackage{tabu}
\usepackage{mathtools}
\usepackage{tikz}
\usepackage{tikz-3dplot}
\usepackage{xcolor}
\usepackage{colortbl}
\usepackage{wasysym}

\begin{document}

\pagesetup{2}{3} 

%%%%% Combining Functions

\card{Consider the following equations: \[f(x)=\frac{x-5}{x-7}\] \[ g(x)=x-5 \]Find the following and state the domain and range for each:\\ $f+g$\\  $f-g$\\ $fg$\\ $\frac{f}{g}$\\ $\frac{g}{f}$\\ $f \circ g$\\ $g \circ f$}{Combining}
\card{Consider the following equations: \[f(x)=\frac{x}{x+5}\] \[ g(x)=\frac{x-1}{x+5} \]Find the following and state the domain and range for each:\\ $f+g$\\  $f-g$\\ $fg$\\ $\frac{f}{g}$\\ $\frac{g}{f}$\\ $f \circ g$\\ $g \circ f$}{Combining}
\card{Consider the following equations: \[f(x)=\frac{1}{\sqrt{2x-5}}\] \[ g(x)=x^2+3 \]Find the following and state the domain and range for each:\\ $f+g$\\  $f-g$\\ $fg$\\ $\frac{f}{g}$\\ $\frac{g}{f}$\\ $f \circ g$\\ $g \circ f$}{Combining}
\card{Consider the following equations: \[f(x)=\frac{1}{2x^2+3}\] \[ g(x)=3x^2-2 \]Find the following and state the domain and range for each:\\ $f+g$\\  $f-g$\\ $fg$\\ $\frac{f}{g}$\\ $\frac{g}{f}$\\ $f \circ g$\\ $g \circ f$}{Combining}
\card{Consider the following equations: \[f(x)=2x+5\] \[ g(x)=\sqrt{x+3} \]Find the following and state the domain and range for each:\\ $f+g$\\  $f-g$\\ $fg$\\ $\frac{f}{g}$\\ $\frac{g}{f}$\\ $f \circ g$\\ $g \circ f$}{Combining}
\card{Consider the following equations: \[f(x)=x^2-2\] \[ g(x)=x+5 \]Find the following and state the domain and range for each:\\ $f+g$\\  $f-g$\\ $fg$\\ $\frac{f}{g}$\\ $\frac{g}{f}$\\ $f \circ g$\\ $g \circ f$}{Combining}
\card{Suppose the value $R(d)$ of $d$ dollars in Euros is given by $R(d)=\frac{6}{7}d$. The cost $P(n)$ in dollars to purchase and ship $n$ purses is given by $P(n)=66n+23$.\\ Write a formula for the cost in Euros to purchase and ship $n$ purses.}{Combining}
\card{The volume $V(r)$ in cubic meters of a spherical balloon with radius $r$ meters is given by $V(r)=\frac{4}{3}\pi r^3$. The radius $W(t)$ after $t$ seconds is given by $W(t)=4t+3$. Write a formula for the volume $M(t)$ of the balloon after $t$ seconds. }{Combining}
\card{The temperature $T(d)$ in degrees Fahrenheit in terms of degrees Celsius is given by $T(d)=\frac{9}{5}d+32$. The temperature $C(v)$ in Celsius in terms of degrees Kelvin is give by $C(v)=c-273$. Write a formula for the temperature in degrees Fahrenheit given a Kelvin temperature. }{Combining}
\card{Consider the following equations: \[f(x)=(x-5)(x+6)\] \[ g(x)=x+2\]Find the following:\\ $(f+g)(3)$\\  $(f-g)(7)$\\ $(fg)(2)$\\ $\left(\frac{f}{g}\right)(-1)$\\ $\left(\frac{g}{f}\right)(-5)$\\ $(f \circ g)(13)$\\ $(g \circ f)(-7)$}{Combining}
\card{Consider the following equations: \[f(x)=(x-1)(x+1)\] \[ g(x)=x-8\]Find the following:\\ $(f+g)(3)$\\  $(f-g)(7)$\\ $(fg)(2)$\\ $\left(\frac{f}{g}\right)(-1)$\\ $\left(\frac{g}{f}\right)(-5)$\\ $(f \circ g)(13)$\\ $(g \circ f)(-7)$}{Combining}
\card{Suppose $H(x)=6\sqrt{x}+3$\\ Find two functions $f$ and $g$ such that $(f \circ g)(x)=H(x)$}{Combining}
\card{Suppose $H(x)=3x^7+1$\\ Find two functions $f$ and $g$ such that $(f \circ g)(x)=H(x)$}{Combining}
\card{Suppose $H(x)=\sqrt[3]{3x+6}$\\ Find two functions $f$ and $g$ such that $(f \circ g)(x)=H(x)$}{Combining}

%%%%% Inverse Functions

\card{A tank is being filled with liquid. The amount of liquid $L$ in liters after $t$ minutes is given by $L(t)=1.25t+73$.\\ Find $L^{-1}$\\ In practical terms what does $L^{-1}$ represent.\\ Find $L^{-1}(34)$.\\ Interpret $L^{-1}(34)$}{Inverses}
\card{Trey is walking. His walking distance $D$ in kilometers from Newbury Heights after $t$ hours of walking is given by $D(t)=13.5-5t$.\\  Find $D^{-1}$\\ In practical terms what does $D^{-1}$ represent.\\ Find $D^{-1}(10)$.\\ Interpret $D^{-1}(10)$ }{Inverses}
\card{Consider the following function: $f(x)=\sqrt{2x-10}$\\ Find the:\\ Domain of $f(x)$\\ Range of $f(x)$\\ $f^{-1}$\\ Domain of $f^{-1}$\\ Range of $f^{-1}$}{Inverses}
\card{Consider the following function: $f(x)=\sqrt{3-x}+8$\\ Find the:\\ Domain of $f(x)$\\ Range of $f(x)$\\ $f^{-1}$\\ Domain of $f^{-1}$\\ Range of $f^{-1}$}{Inverses}
\card{Consider the following function: $f(x)=9-x^3$\\ Find the:\\ Domain of $f(x)$\\ Range of $f(x)$\\ $f^{-1}$\\ Domain of $f^{-1}$\\ Range of $f^{-1}$}{Inverses}
\card{Consider the following function: $f(x)=(x-6)^3$\\ Find the:\\ Domain of $f(x)$\\ Range of $f(x)$\\ $f^{-1}$\\ Domain of $f^{-1}$\\ Range of $f^{-1}$}{Inverses}
\card{Consider the following function: $f(x)=\sqrt[3]{x+4}+8$\\ Find the:\\ Domain of $f(x)$\\ Range of $f(x)$\\ $f^{-1}$\\ Domain of $f^{-1}$\\ Range of $f^{-1}$}{Inverses}
\card{Consider the following function: $f(x)=\sqrt[3]{10-x}+8$\\ Find the:\\ Domain of $f(x)$\\ Range of $f(x)$\\ $f^{-1}$\\ Domain of $f^{-1}$\\ Range of $f^{-1}$}{Inverses}
\card{Consider the following function: $f(x)=\frac{x-8}{x-7}$\\ Find the inverse of $f(x)$}{Inverses}
\card{Consider the following function: $f(x)=\frac{-x-4}{x+12}$\\ Find the inverse of $f(x)$}{Inverses}


%%%%% Quadratic Functions
\card{Find the zeros of:\\ $v^2=-13v$\\ $y^2-11y+30$\\ $9v^2-30v+25$\\ $2y^2-15y+43=(y-2)^2$\\ $(w^2+6w+8)(2w-18)$}{Quadratics}
\card{Find the zeros of:\\ $6v^2=-12v$\\ $y^2=y-2$\\ $5w^2-14w-3$\\ $2w^2+8w+42=(w+7)^2$\\ $(4u+12)(u^2-25)$ }{Quadratics}
\card{Write the equation for a quadratic function:\\ that has roots of -2 and -5 with leading coefficient 2\\ has a vertex at $(-1,1)$ and passes through the point $(-3,-7)$}{Quadratics}
\card{Write the equation for a quadratic function:\\ that has roots of -3 and 2 with leading coefficient 4\\ has a vertex at $(-3,5)$ and passes through the point $(-5,13)$ }{Quadratics}
\card{The area of a rectangle is 35 square feet and its length is 3 more than 2 times its width. What are the dimensions of the rectangle?}{Quadratics}
\card{The area of a rectangle is 44 square feet and its length is 3 feet less than twice the width. What are the dimensions of the rectangle?}{Quadratics}
\card{Find the zeros of:\\ $2x^2+6x-3$\\ $4x^2-6x+1$\\ $9x^2+5x-1$\\ $9x^2-x-1$ }{Quadratics}
\card{A model rocket is launched and its height can be modeled by the function $h=50t-5t^2$\\ Find:\\ all values for which the rocket is at 20ft\\ the maximum height the rocket reaches\\ the time the rocket is at is maximum height\\ the initial height from which the rocket was launched\\ when the rocket lands}{Quadratics}
\card{A ball is thrown into the air and its height can be modeled by the function $h=3+13t-5t^2$\\ Find:\\ all values for which the ball is at 10 meters\\ the maximum height the ball reaches\\ the time the ball is at is maximum height\\ the initial height from which the ball was thrown\\ when the ball hits the ground}{Quadratics}
\card{When an apple orchard owner plants 65 trees on an acre of ground, he gets an average yield of 1500 apples per tree per year. For each additional tree planted per acre, the annual yield per tree drops by 20 apples. Let $x$ represent the number of additional trees above 65.\\ Write a formula that represents the number of apples the orchard produces every year\\ How many trees should be planted to maximize the crop?\\ What is the maximum crop?}{Quadratics}
\card{When an apple orchard owner plants 70 trees on an acre of ground, he gets an average yield of 1400 apples per tree per year. For each additional tree planted per acre, the annual yield per tree drops by 10 apples. Let $x$ represent the number of additional trees above 70.\\ Write a formula that represents the number of apples the orchard produces every year\\ How many trees should be planted to maximize the crop?\\ What is the maximum crop?}{Quadratics}
\card{The manager of an 80-unit apartment complex is trying to decide what rent to charge each month. At a rent of \$800, all 80 units are full. On average, one additional unit will become vacant for each \$25 increase in rent.\\ Write an equation that represents the revenue with respect to $x$ where $x$ is the price charged for rent.\\ How much should be charged to maximize revenue?\\ What is the maximum monthly revenue?}{Quadratics}




%%%%% Polynomial Functions 
\card{For the following polynomials state the degree, leading coefficient, end behavior and find the zeros:\\ $20w^3-w^8+8w^6+18w$\\ $-v+7v^5-15v^2-7v^4$\\ $-7w-w^7+9$}{Polynomial}
\card{For the following polynomials state the degree, leading coefficient, end behavior and find the zeros:\\ $16w^3-w^7+8w^6+18w$\\ $-v+7v^6-15v^2-7v^4$\\ $-8w-w^7+9$}{Polynomials}
\card{According to their website, American Airlines limits the size of checked baggage in such a way that the length+width+height of the bag can be no more than 72 inches. Suppose a piece of luggage has length and width equal to $x$ inches.\\ Write a function to express the volume of the luggage as described above.\\ What is an appropriate domain for the function.\\ Find the maximum volume for such a piece of luggage.\\ What dimensions of the suitcase maximize volume.}{Polynomials}
\card{According to their website, American Airlines limits the size of checked baggage in such a way that the length+width+height of the bag can be no more than 45 inches. Suppose a piece of luggage has length and width equal to $x$ inches.\\ Write a function to express the volume of the luggage as described above.\\ What is an appropriate domain for the function.\\ Find the maximum volume for such a piece of luggage.\\ What dimensions of the suitcase maximize volume.}{Polynomials}
\card{The function $f(x)=0.67x^3-4.52x^2+14.83x-8.21$ approximates the number of Facebook friends a person has after the first 15 days of creating their Facebook account. Assume the value of $x$ represents the start of the 15th day.\\ During the middle of which day would a person have 400 Facebook friends?}{Polynomials}
\card{The function $f(x)=0.37x^3-2.52x^2+17.83x-10.21$ approximates the number of times a person has swiped right after the first 20 days of creating their Tinder account. Assume the value of $x$ represents the start of the 20th day.\\ During the middle of which day would a person have swiped right 100 times?}{Polynomials}
 
 
 
 
 
 
 
 
 
 
\end{document}
















