\documentclass{beamer}

\usepackage{amsmath, epsfig, subfigure, psfrag, amssymb, amsfonts, array, amsthm}
\usepackage{enumerate}
\usepackage{tikz}
	\usetikzlibrary{calc, shapes.geometric, positioning, shapes, decorations, arrows, patterns}
\usepackage{wasysym}
 \usetikzlibrary{decorations.markings}

\mode<presentation>

\usetheme{Pittsburgh}
\usecolortheme{beaver}
%\useoutertheme{split}

\setbeamertemplate{footline}{}%[split, frame number]
\setbeamertemplate{enumerate items}[default]
\setbeamertemplate{itemize items}[circle]

\setbeamersize{text margin left=6mm}
\setbeamersize{text margin right=6mm}
\setbeamersize{sidebar width right=0mm}
\setbeamersize{sidebar width left=0mm}
\setbeamertemplate{navigation symbols}{}

\newtheorem{comments}{Comments}
\newtheorem{question}{Question}
\newtheorem{goal}{Goal}
\newtheorem{remark}{Remark}
\newtheorem{proposition}{Proposition}
\newtheorem{conjecture}{Conjecture}

% \newcommand*\oldmacro{}
% \let\oldmacro\insertshorttitle
% \renewcommand*\insertshorttitle{
% \oldmacro\hfill
% \insertframenumber\,/\,\inserttotalframenumber}

\def\tsimkappa{\sim_{\kappa}}
\def\r{\vec{r}}
\newcommand{\w}{\overline{w}}
\DeclareMathOperator{\sgn}{sgn}
\DeclareMathOperator{\FC}{FC}
\newcommand{\C}{\widetilde{C}}
\DeclareMathOperator{\Sym}{Sym}
\DeclareMathOperator{\supp}{supp}
\newcommand{\LD}{\mathcal{L}}
\newcommand{\RD}{\mathcal{R}}


%% ----------------------------------------------------------------------  

\begin{document}

\def\newblock{\hskip .11em plus .33em minus .07em}

\title[Math 112 Final Review]
{\textbf{Math 112 Final Review}}
\author[T.M.~Laird]{Taryn Laird}
\institute[UA]{University of Arizona\\
Department of Mathematics}

\vspace{1em}

\date[UofA]{\textbf{Math 112}\\
May 2, 2018}

\frame{\titlepage}

%% ----------------------------------------------------------------------



%\begin{frame}{\textbf{Question 1}}
%	Solve the equation.
%\[ 2(x+3)-5=3x+10\]
%%	\pause
%%	
%%	Answer: $-9$
%\end{frame}
%
%\begin{frame}{\textbf{Question 2}}
%	Solve the equation.
%\[\frac{2}{x-5} + \frac{3}{x} = \frac{5}{x^2-5x}\]
%	
%%	\pause
%%	
%%	Answer:$x=-2$
%\end{frame}
%
%\begin{frame}{\textbf{Question 3}}
%	Determine the values of $x$ and $y$ that solve the following system of equations.
%\begin{center}
%\begin{align*}
%	3x+6y &=22\\
%	-x+2y &= 10\\
%\end{align*}
%\end{center}
%
%	\bigskip
%	
%%	\pause
%%	
%%	Answer: Between 13 and 17
%\end{frame}

\begin{frame}{\textbf{Question 1}}
	A couple invests \$4,000 into an apiary. On average each pint of honey the apiary produces costs \$2.76 to produces and sells for \$10.20 per pint. How many pints of honey does the couple need to sell in order to break even?
	\bigskip
	
%	\pause
%	
%	Answer: Between 2.5 and 3 mph
\end{frame}

%\begin{frame}{\textbf{Question 5}}
%	A shop owner wants to mix high quality tea leaves that cost \$4.95 per pound with lower quality tea leaves that cost \$2.55 per pound to obtain 30 pounds of tea blend that costs \$3.10 per pound. How much of each type of tea should he add to his blend?
%	\bigskip
%	
%%	\pause
%%	
%%	Answer: 21 hours
%\end{frame}

%\begin{frame}{\textbf{Question 6}}
%	Calculate the discriminant and state how many solutions the following quadratic equation has.
%\[ 3x^2-5x+1\]
%
%	
%	\bigskip
%	
%%	\pause
%%	
%%	Answer: $-1+\sqrt{10}$
%\end{frame}

\begin{frame}{\textbf{Question 2}}
	Solve for $x$.
\[(x-1)(x+10)=16\]
	
	\bigskip
	
%	\pause
%	
%	Answer: 75 mph
\end{frame}

\begin{frame}{\textbf{Question 3}}
	 Consider the quadratic function $f(x)= 3x^2-13-10$. What is the vertex of $f(x)$?
	
	\bigskip
	
%	\pause
%	
%	Answer: $y=\frac{2}{3}x+1$
\end{frame}

\begin{frame}{\textbf{Question 4}}
	A toy rocket is launched vertically in the air from a 7-foot launching platform with an initial velocity of 40 meters per second. If the equation modeling the height of the rocket is given by $h(t)=-4.9t^2+v_0t+h_0$ where $v_0$ is the initial velocity, and $h_0$ is the initial height, what is the maximum height reached by the rocket?

	
%	\pause
%	
%	Answer: 5.
\end{frame}

%\begin{frame}{\textbf{Question 10}}
%	Farmer Ed has $8,000$ meters of fencing, and wants to enclose a rectangular field, that borders a river on one side. If Farmer Ed, doesn't enclose the side of the field that borders the river, what are the dimensions of the field that maximizes the area?
%	
%	\bigskip
%	
%%	\pause
%%	
%%	Answer: $\frac{f(h+4)-f(4)}{h}=3h+19$.
%\end{frame}

%\begin{frame}{\textbf{Question 11}}
%	Find the domain of the equation $f(x)=\sqrt{2-5x}$\\
%	
%	\bigskip
%	
%%	\pause
%%	
%%	Answer: $\left(-\infty, \frac{2}{5}\right]$
%\end{frame}

%\begin{frame}{\textbf{Question 12}}
%	Determine which of the following represent $y$ as a function of $x$. Explain why the equation is or is not a function:
%	\begin{itemize}
%		\item $x^2+y^2-4 =0$
%		\item $x^2+y^3=-27$
%		\item $xy-y^4+4=8$
%	\end{itemize}	
%%	\pause
%%	
%%	Answer: Less than \$20,000.
%\end{frame}

%\begin{frame}{\textbf{Question 13}}
%	Write an equation that has a domain of all real numbers except $x=-3$ and $x=5$.	
%\end{frame}
%
%\begin{frame}{\textbf{Question 14}}
%	Determine an equation for the polynomial that has zeros at $x=0, -1, 1, \sqrt{3}, -\sqrt{3}$ and passes through the point $(4,10)$.\\
%	
%	\bigskip
%	
%%	\pause
%%	
%%	Answer: $x=0, x=1$
%\end{frame}

\begin{frame}{\textbf{Question 5}}
	Consider the polynomial function $f(x)=3(x-5)^2(x^2+2)^3(x+3)^4$. What is the end behavior of the graph?
	
	\bigskip
	
%	\pause
%	
%	Answer: $f(-1)=-2$ and $f(2)=5$
\end{frame}

\begin{frame}{\textbf{Question 6}}
	Consider the rational function: \[f(x)=\frac{x^2-4}{(x^2-x-6)}\]	 What is the domain of $f(x)$? 
	 \bigskip
%	 \pause
%	 
%	 Answer: $(0, 3)$, $(\frac{-3}{2}, 0)$
\end{frame}

%\begin{frame}{\textbf{Question 17}}
%	Consider the rational function: \[f(x)=\frac{x^2-4}{(x^2-x-6)}\]. What are the removable discontinuities of $f(x)$?
%	\bigskip
%	
%%	Answer: $(-4, -5)$
%\end{frame}

\begin{frame}{\textbf{Question 7}}
	Consider the rational function: \[f(x)=\frac{x^2-4}{(x^2-x-6)}\] What is the horizontal asymptote of $f(x)$?
	\bigskip	
%	\pause
%	
%	Answer: $(h \circ f)(x)=\sqrt{x^2-9}$
\end{frame}

\begin{frame}{\textbf{Question 8}}
	Consider the rational function: \[f(x)=\frac{x-4}{(x^2+x-6)}\] What are the intercepts of $f(x)$?
	
	\bigskip
	
%	\pause
%	
%	Answer: $f^{-1}(x)=\frac{-3x-1}{2x-1}$
\end{frame}

\begin{frame}{\textbf{Question 9}}
	Find a formula for the parabola whose vertex is at $(-2,-1)$ and passes through the point $(0,13)$.\\
	\bigskip
	
%	\pause
%	Answer: $y=3(x+2)^2-1$
\end{frame}

%\begin{frame}{\textbf{Question 20}}
%	Find a formula for the parabola whose vertex is at $(-2,-1)$ and passes through the point $(0,13)$.\\
%	\bigskip
%	
%%	\pause
%%	Answer: $y=3(x+2)^2-1$
%\end{frame}

\begin{frame}{\textbf{Question 10}}
	Consider the following exponential function: \[f(x)=3 \cdot \left(\frac{5}{3}\right)^{-2}+3\]. List the transformations of the base graph of $f(x)$.
	\bigskip
	
%	\pause
%	Answer: $y=3(x+2)^2-1$
\end{frame}

\begin{frame}{\textbf{Question 11}}
	Consider the following exponential function: \[f(x)=3 \cdot \left(\frac{5}{3}\right)^{-x}+3\]. What is the asymptote of $f(x)$?
	\bigskip
	
%	\pause
%	Answer: $y=3(x+2)^x-1$
\end{frame}

\begin{frame}{\textbf{Question 12}}
	Solve for $x$: \[\frac{e^{x+5}}{x^{3x}}=e^{x-1}\]
	\bigskip
	
%	\pause
%	Answer: $y=3(x+2)^2-1$
\end{frame}

\begin{frame}{\textbf{Question 13}}
	Consider the equation: \[f(x)=\log_3(x+11)\] (1) What is the domain of $f(x)$?
	\bigskip
	
%	\pause
%	Answer: $y=3(x+2)^2-1$
\end{frame}

\begin{frame}{\textbf{Question 14}}
	Suppose 128 ounces of a radioactive substance exponentially decays to 28 ounces in 6 hours. What is the half-life of the substance?
	\bigskip
	
%	\pause
%	Answer: $y=3(x+2)^2-1$
\end{frame}


\end{document}
